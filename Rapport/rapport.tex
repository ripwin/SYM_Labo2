% -----------------------------------------------------------------------
% --- DOCUMENTS ---
% -----------------------------------------------------------------------
\documentclass[francais,12pt]{article}
\usepackage[utf8]{inputenc}
\usepackage{ae, pslatex}
\usepackage[french]{babel}
\selectlanguage{french} 

\usepackage{mathtools}
\usepackage{amssymb}
\usepackage{pgfplots}
\usepackage{caption}
\usepackage{hyperref}
\hypersetup{
	colorlinks=true,
	linkcolor=blue,
	filecolor=magenta,      
	urlcolor=cyan,
}

\usepackage{titlesec}
\usepackage{color}
\usepackage{colortbl}

\usepackage{hhline,tabu}

\frenchbsetup{StandardLists=true}

% -----------------------------------------------------------------------
% --- CODE JAVA ---
% -----------------------------------------------------------------------
\usepackage{listings} % pour afficher du code
\definecolor{mauve}{rgb}{0.472,0.035,0.218}
\definecolor{darkGreen}{rgb}{0.0429,0.601,0.0117}
\definecolor{antiFlashWite}{rgb}{0.95,0.95,0.96}


\lstdefinelanguage{Java}{
	morekeywords={typeof, new, true, false, catch, function, return, null, catch, switch, var, if, in, while, do, else, case, break, let,this},
	morecomment=[s]{/*}{*/},
	morecomment=[l]//,
	morestring=[b]",
	morestring=[b]',
	morestring=[s]{/[}{/;}
}

\lstdefinestyle{javaCode}
{
	% language related
	language=Java,
	keywordstyle=\color{blue},
	commentstyle=\color{green},
	stringstyle=\color{mauve},
	basicstyle=\footnotesize\ttfamily, % Standardschrift
	numbers=left,               % Ort der Zeilennummern
	numberstyle=\tiny,          % Stil der Zeilennummern
	stepnumber=2,               % Abstand zwischen den Zeilennummern
	numbersep=5pt,              % Abstand der Nummern zum Text
	tabsize=4,                  % Groesse von Tabs
	extendedchars=true,         %
	breaklines=true,            % Zeilen werden Umgebrochen
	keywordstyle=\color{blue}\bfseries,
	frame=b,
	showspaces=false,           % Leerzeichen anzeigen ?
	showtabs=false,             % Tabs anzeigen ?
	xleftmargin=17pt,
	framexleftmargin=17pt,
	framexrightmargin=5pt,
	framexbottommargin=4pt,
	backgroundcolor=\color{antiFlashWite},
	showstringspaces=false,      % Leerzeichen in Strings anzeigen ?
}

% -----------------------------------------------------------------------
% --- MARGES ---sp
% -----------------------------------------------------------------------
\usepackage{vmargin}
\setpapersize{A4}
\setmarginsrb{60pt}{50pt}{60pt}{25pt}{15pt}{25pt}{15pt}{25pt}

% -----------------------------------------------------------------------
% --- EN-TETE ET PIED DE PAGE ---
% -----------------------------------------------------------------------
\usepackage{fancyhdr}
\usepackage{lastpage}
\pagestyle{fancy}

\fancyhead[L]{SYM - Systèmes mobiles}
\fancyhead[R]{IL - TIC - HEIG-VD \\ Automne 2017}
\fancyfoot[C]{\thepage{}}

\title{Systèmes mobiles \\ Laboratoire n\textordmasculine2 : Protocoles applicatifs}
\author{Mathieu Monteverde, Sathiya Kirushnapillai, Michela Zucca}
\date{Automne 2017}

\titlespacing\section{0pt}{12pt plus 4pt minus 2pt}{0pt plus 2pt minus 2pt}
\titlespacing\subsection{0pt}{12pt plus 4pt minus 2pt}{0pt plus 2pt minus 2pt}
\titlespacing\subsubsection{0pt}{12pt plus 4pt minus 2pt}{0pt plus 2pt minus 2pt}


% ***********************************************************************
% *** DOCUMENT PRINCIPAL ***
% ***********************************************************************
\begin{document}
	
	\maketitle
	
	\setlength{\parskip}{1em}
	
	\section*{Manipulations}
	
	\subsection*{Manipulation 3.2 - Transmission différée}
	
	Pour cette manipulation nous avons créé la classe \textbf{DelayedTransmission} qui présente les mêmes méthodes que la classe \textbf{AsyncTransmission}, mais qui s'occupe également d'envoyer à nouveau les requêtes qui n'ont pas pu être transmises en cas de perte de connexion. 
	
	\subsubsection*{Détection de l'erreur}
	Les méthodes de la classe \textbf{OkHttpClient} permettent de spécifier une méthode de callback lorsqu'une erreur de transmission est survenue. Grâce à cela, lorsqu'elle est notifiée d'une telle erreur, la classe \textbf{DelayedTransmission} ajoute la requête fautive dans une liste de requêtes qu'il faudra tenter d'envoyer à nouveau. 
	
	\subsubsection*{Plannification d'une nouvelle tentative d'envoi}
	Lorsqu'une requête à échoué, elle est donc ajoutée à cette liste et une nouvelle tentative est planifiée et sera exécutée dix secondes plus tard. La classe garde trace également de ces demandes et fait en sorte que de nouvelles tentatives ne soient pas planifiées si il y a déjà une tentative en attente d'être exécutée.
	
	\subsubsection*{Exécution d'une tentative}
	Lorsqu'une tentative est exécutée, elle essaie d'envoyer toutes les requêtes présentes dans la liste de requêtes en échec et vide cette liste après coup. En suivant cette logique, si la connexion est revenue entretemps, les envois s'effectueront avec succès, En revanche si un problème de connexion persiste, une nouvelle tentative sera planifiée lorsque l'objet sera notifié d'une erreur d'envoi. 
	
	\subsubsection*{Exclusion mutuelle}
	Toutes ces actions sont exécutées en exclusion mutuelle afin d'éviter d'ajouter une requête dans la liste alors que la classe est en train d'effectuer une nouvelle tentative. 
	
	\subsubsection*{Tests et affichage des résultats}
	Nous avons testé cette classe en coupant le WIFI avant d'envoyer les requêtes, et en le rallumant une dizaine de secondes plus tard. Nous avons ensuite essayé en coupant le WIFI juste après avoir cliqué sur le bouton pour envoyer les requêtes. Dans les deux cas, les requêtes ont été envoyées à nouveau et les réponses nous sont parvenues. 
	
	Pour l'affichage des réponses du serveur, nous affichons la première ligne dans l'interface utilisateur et la réponse au complet dans la console. Chaque transmission possède un numéro incrémental pour mieux les distinguer.
	
	\subsubsection*{Limitations}
	Les limitations de cette solutions sont évidentes. La liste de requêtes en échec sont contenues en mémoire dans l'instance de l'activité en question. Si l'utilisateur quitte l'activité, ou quitte l'application avant que les requêtes soient effectuées avec succès, la liste de requêtes qui ont échouées seront perdues. 
	
	Cela peut poser problème dans de très nombreux domaines. Une solution possible serait de sauvegarder les informations en mémoire persistante lorsque l'activité en question est détruite et de lire ces informations lorsqu'on démarre à nouveau l'application ou l'activité concernée. Cela deviendra vite difficile de garder un logiciel modulable cela dit.
	
	\subsubsection*{JobScheduler}
	Depuis Android 5.0 (Lollipop), la class abstraite JobScheduler est proposée et permet de planifier des actions à effectuer. Apparemment elle permet de planifier des actions non seulement à certains intervalles de temps, mais également lorsque certaines conditions sont remplies. On pourrait donc utiliser cette fonctionnalité pour gérer nos requêtes qui ont échoué et demander qu'elles soient envoyées lorsque l'appareil possède à nouveau une connexion internet. 
	
	\textbf{Sources (consultées le 19.11.2017) :}
	
	https://josiassena.com/the-jobscheduler-on-android/ \\
	https://medium.com/google-developers/scheduling-jobs-like-a-pro-with-jobscheduler-286ef8510129 \\
	https://josiassena.com/the-jobscheduler-on-android/ \\
	
	
	\section*{Réponse aux questions}
	
	\subsection*{Questions 1 - Traitement des erreurs}
	
	Les interfaces \textit{AsyncSendRequest} et \textit{CommunicationEventListener} utilisées au point 3.1 restent très (certainement trop) simples pour être utilisables dans une vrai application : que se passe-t-il si le serveur n'est pas joignable dans l'immédiat ou s'il retourne un code HTTP d'erreur ? Vous pouvez par exemple proposer une nouvelle version de ces deux interfaces pour vous aider à illustrer votre réponse.
	
	{\color[rgb]{0,0.5,0.23}\textbf{Réponse}}
	
	Pour répondre à cette question il faut d’abord spécifier le contexte de transmission.
	
	\subsubsection*{Dans le cas d’une transmission synchrone}
	Il faudrait modifier la signature de la méthode sendRequest pour inclure des levées d’exceptions plus précises. 
	
	La méthode SendRequest devrait lever toutes les exceptions relatives à des mauvaises données reçues avant tous envois au serveur et permettre à l’application de modifier la structure de sa requête. Et toutes les exceptions nécessaires pour réagir en cas d'erreur en réponse du serveur. Une réponse sans échec utiliserait le type de retour de la méthode pour être transmise à l'application. 
	
	Par exemple une requête nécessitant une authentification qui n’aura pas pu être transmise avant que le token d’authentification périme ne pourra pas aboutir. Dans ce cas il faudra lever une exception pour prévenir l’application de cet échec et lui permettre de demander une nouvelle connexion à l’utilisateur et une fois celle-ci faite, renvoyer la requête échouée.
	
	Évidemment cela implique d’utiliser un système de sauvegarde, par exemple un cache, une base de données, ou autres, pour conserver les requêtes qui n’ont pu être traitées.  Et encore décidé de quels informations seront conservées pour que l’application puisse les réutiliser.
	
	On pourrait proposer une méthode comme ceci : 
	\begin{itemize}
		\item sendRequest
		\begin{itemize}
			\item IllegalArgumentException : les paramètres reçu ne sont pas conforme à ceux attendus par la méthode.
			\item IOException : problème de communication sur le réseau
			\item HTTPException : pour traiter les différents cas lié au code d'erreur HTTP. Exemple : code 401 Unauthorized, lorsque le serveur nécessite une connexion.
		\end{itemize}
	\end{itemize}
	
	\begin{lstlisting}[style=javaCode]
		public String sendRequest(String request, String url) throws IllegalArgumentException, IOException, HTTPException
	\end{lstlisting}
	
	\subsubsection*{Dans le cas d’une transmission asynchrone}
	Les transmissions asynchrones utilisent le concept de callback. C’est la méthode que nous avons utilisée dans notre labo. Pour pouvoir transmettre l’erreur à l’application principale il faut ajouter une méthode error à l’interface CommunicationEventListener qui sera appelée dans le cas d’une transaction échouée. La classe AsyncTransmission pourra retransmettre les informations nécessaires concernant l’erreur rencontrée au travers de cette méthode. 
	
	Ainsi, comme pour l’autre solution, nous offrons la possibilité de réagir en cas d’erreur et de prendre les décisions utiles à l’application. 
	
	\subsection*{Questions 2 - Authentification}
	Si une authentification par le serveur est requise, peut-ont utiliser un protocole asynchrone ? Quelles seraient les restrictions ? Peut-on utiliser une transmission différée ?
	
	
	{\color[rgb]{0,0.5,0.23}\textbf{Réponse}}
	
	L'authentification et l'utilisation d'une transmission différée n'est pas un problème avec les transmissions asynchrones. 
	
	Les restrictions peuvent être diverses. Comme soulevé à la question 1, si on lève des exceptions précisent permettant à l'application principale de gérer les erreurs du protocole, serveur ou client, il est possible de parer au problème d'identification. 
	
	Par exemple si on utilise un token pour identifier le client et que celui-ci n'est plus valide lors de l'envoi de la requête, on signale à l'application le problème au travers d'une exception. L'application pourra par exemple notifié l'utilisateur qu'une authentification est requise, et une fois celle-ci faite regénérer la requête. Cela implique de sauvegarder les requêtes n'ayant pas aboutie. 
	
	Il est également possible de mettre en place un système d'échange de clé pour authentifier le client. A noter que la clé ne doit pas être enregistré dans l'application avant son installation, sans quoi elle serait vulnérable.
	
	\subsection*{Questions 3 - Threads concurrents}
	Lors de l'utilisation de protocoles asynchrones, c'est généralement deux threads différents qui se préoccupent de la préparation, de l'envoi, de la réception et du traitement des données. Quels problèmes cela peut-il poser ?
	
	{\color[rgb]{0,0.5,0.23}\textbf{Réponse}}
	
	
	Un thread autre que le thread UI ne doit pas modifier la vue. Si d'autres threads pouvaient modifier la vue on aurait des problèmes de concurrences. Imaginons 2 threads s'occupent de gérer de A à Z la communication, préparation, envoi, réception et mis à jour de l’UI. Si le thread A commencé à modifier la vue et se fait préempter, le thread B commence à son tour  à modifier la vue en partie et se fait préempter.. Il n'y a plus aucune cohérences entres les données. Par contre si seul le thread de l'UI à droit de modifier l’UI alors on assure la cohérence des données. Car quelques soit les requêtes en attente de traitement, elles seront traitées séquentiellement les unes après les autres.
	
	Un autre problème moins grave est que de base il n'y a pas de priorité sur les tâches à exécuter. Ainsi il est possible qu'une action "mineure" passe devant une demande d'authentification. Par exemple l'affichage d'une image. Pour parer à ce problème il est possible de créer un ordre de priorité sur les communicationEventListener reçu par le protocole de communication. 

	
	\subsection*{Questions 4 - Écriture différée}    
	Lorsque l'on implémente l'écriture différée, il arrive que l'on ait soudainement plusieurs transmissions en attente qui deviennent possibles simultanément. Comment implémenter proprement cette situation (sans réalisation pratique) ? Voici deux possibilités :
	\begin{itemize}
		\item Effectuer une connexion par transmission différée
		\item Multiplexer toutes les connexions vers un même serveur en une seule connexion de transport. Dans ce dernier cas, comment implémenter le protocole applicatif, quels avantages peut-on espérer de ce multiplexage, et surtout, comment doit-on planifier les réponses du serveur lorsque ces dernières s'avèrent nécessaires ?
	\end{itemize}
	
	Comparer les deux techniques ( et éventuellement d'autres que vous pourriez imaginer) et discuter des avantages et inconvénients respectifs.
	
	{\color[rgb]{0,0.5,0.23}\textbf{Réponse}}
	
	\subsubsection*{Effectuer une connexion par transmission différée}
	Lorsqu'on décide d'effectuer une connexion par transmission, cela a pour avantage de ne pas changer le protocole de base utilisé pour les transmissions en temps normal. Le serveur pourra traiter chaque transmission comme si elle n'était pas différée. 
	
	En revanche cela peut poser des problèmes au niveau des débits de connexion. En effet envoyer beaucoup de transmission presque simultanément peut utiliser beaucoup de bande passante. Sur mobile, c'est embêtant. 
	\subsubsection*{Effectuer une seule connexion pour toutes les transmissions différées}
	Si on décide de regrouper toutes les transmissions sur une seule connexion, cela peut permettre d'éviter un échange inutile d'information. Par exemple, en HTPP, cela permet de n'avoir qu'une fois la phase de Handshake pour N transmissions, contrairement à N fois la phase de Handshake pour la solution précédente. 
	
	En revanche, cela demande d'adapter le protocole pour que le serveur puisse savoir qu'il s'agit d'une transmission multiple. Il faut également prévoir une solution pour que le serveur puisse retourner les réponses éventuelles en donnant un moyen d'identifier la transmission correspondante pour l'application client. Par exemple chaque transmission aurait son identifiant unique et la réponse du serveur marquerait chaque réponse avec l'identifiant de la requête concernée.
	
	Les deux solutions présentent leurs avantages et inconvénients. La première est la solution la plus simple du point de vue implémentation, mais souffre de défauts de performance et d'optimisation. La seconde en revanche permet d'optimiser la communication mais demande de développer un protocole de communication dédié et complexifie beaucoup la gestion de requêtes et de réponses entre le serveur et le client.
	
	
	\subsection*{Questions 5 - Transmission d'objets}  
	\begin{enumerate}
		\item Quel inconvénient y a-t'il à utiliser une infrastructure de type REST/JSON n'offrant aucun service de validation (DTD, XML-schéma, WSDL) par rapport à une infrastructure comme SOAP offrant ces possibilités ? Y a-t'il en revanche des avantages que vous pouvez citer ?
		\item L'utilisation d'un mécanisme comme Protocol Buffer est-elle compatible avec une architecture basée sur HTTP? Veuillez discuter des éventuelles avantages ou limitations par rapport à un protocole basé sur JSON ou XML.
	\end{enumerate} 
	
	{\color[rgb]{0,0.5,0.23}\textbf{Réponse}}
	
    Ces deux infrastructures ne sont pas vraiment comparable et ne s'utilisent pas pour la même chose. SOAP est principalement utilisé dans les environnements distribués. Ce protocole est standardisé et permet facilement la validations des données reçus et la gestion des erreurs. Cependant, SOAP est complexe et relativement volumineux.
    
    REST est plus dans la philosophie du Web simple, léger, rapide et facile de prise en main. Cependant, il requière HTTP pour la communication alors que ce n'est pas le cas de SOAP qui est indépendant à toutes plateformes. Il effectue uniquement des communications point à point. Et enfin il est beaucoup plus permissif et donc enclin aux erreurs.
    
    L'utilisation d'un mécanisme comme Protocol Buffers est compatible avec une architecture basée sur HTTP. Les avantages à utiliser Protocol Buffers au lieu de XML sont la simplicité, plus léger et plus rapide. Et le développement s'effectue plus facilement. Et contrairement à Json, il permet la validation des données facilement à l'aide des fichiers de description.
    
    Les désavantage sont que les données ne sont plus lisible par un humain (Sérialisation sous forme binaire). Il y a une très petite communauté et un manque énorme du support et de ressources.
	
	\subsection*{Question 6 - Transmission compressée}
	Quel gain peut-on constater en moyenne sur des fichiers texte (xml et json sont aussi du texte) en utilisant de la compression du point 3.4 ? Vous comparerez vos résultats par rapport au gain théorique d'une compression DEFLATE, vous enverrez plusieurs tailles de contenu pour comparer.
	
	{\color[rgb]{0,0.5,0.23}\textbf{Réponse}}
	
	Réponse
	
	
\end{document}

