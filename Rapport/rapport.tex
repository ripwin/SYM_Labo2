% -----------------------------------------------------------------------
% --- DOCUMENTS ---
% -----------------------------------------------------------------------
\documentclass[francais,12pt]{article}
\usepackage[utf8]{inputenc}
\usepackage{ae, pslatex}
\usepackage[french]{babel}
\selectlanguage{french} 

\usepackage{mathtools}
\usepackage{amssymb}
\usepackage{pgfplots}
\usepackage{caption}
\usepackage{hyperref}
\hypersetup{
	colorlinks=true,
	linkcolor=blue,
	filecolor=magenta,      
	urlcolor=cyan,
}

\usepackage{titlesec}
\usepackage{color}
\usepackage{colortbl}

\usepackage{hhline,tabu}

\frenchbsetup{StandardLists=true}

% -----------------------------------------------------------------------
% --- CODE JAVA ---
% -----------------------------------------------------------------------
\usepackage{listings} % pour afficher du code
\definecolor{mauve}{rgb}{0.472,0.035,0.218}
\definecolor{darkGreen}{rgb}{0.0429,0.601,0.0117}
\definecolor{antiFlashWite}{rgb}{0.95,0.95,0.96}


\lstdefinelanguage{Java}{
	morekeywords={typeof, new, true, false, catch, function, return, null, catch, switch, var, if, in, while, do, else, case, break, let,this},
	morecomment=[s]{/*}{*/},
	morecomment=[l]//,
	morestring=[b]",
	morestring=[b]',
	morestring=[s]{/[}{/;}
}

\lstdefinestyle{javaCode}
{
	% language related
	language=Java,
	keywordstyle=\color{blue},
	commentstyle=\color{green},
	stringstyle=\color{mauve},
	basicstyle=\footnotesize\ttfamily, % Standardschrift
	numbers=left,               % Ort der Zeilennummern
	numberstyle=\tiny,          % Stil der Zeilennummern
	stepnumber=2,               % Abstand zwischen den Zeilennummern
	numbersep=5pt,              % Abstand der Nummern zum Text
	tabsize=4,                  % Groesse von Tabs
	extendedchars=true,         %
	breaklines=true,            % Zeilen werden Umgebrochen
	keywordstyle=\color{blue}\bfseries,
	frame=b,
	showspaces=false,           % Leerzeichen anzeigen ?
	showtabs=false,             % Tabs anzeigen ?
	xleftmargin=17pt,
	framexleftmargin=17pt,
	framexrightmargin=5pt,
	framexbottommargin=4pt,
	backgroundcolor=\color{antiFlashWite},
	showstringspaces=false,      % Leerzeichen in Strings anzeigen ?
}

% -----------------------------------------------------------------------
% --- MARGES ---sp
% -----------------------------------------------------------------------
\usepackage{vmargin}
\setpapersize{A4}
\setmarginsrb{60pt}{50pt}{60pt}{25pt}{15pt}{25pt}{15pt}{25pt}

% -----------------------------------------------------------------------
% --- EN-TETE ET PIED DE PAGE ---
% -----------------------------------------------------------------------
\usepackage{fancyhdr}
\usepackage{lastpage}
\pagestyle{fancy}

\fancyhead[L]{SYM - Systèmes mobiles}
\fancyhead[R]{IL - TIC - HEIG-VD \\ Automne 2017}
\fancyfoot[C]{\thepage{}}

\title{Systèmes mobiles \\ Laboratoire n\textordmasculine2 : Protocoles applicatifs}
\author{Mathieu Monteverde, Sathiya Kirushnapillai, Michela Zucca}
\date{Automne 2017}

\titlespacing\section{0pt}{12pt plus 4pt minus 2pt}{0pt plus 2pt minus 2pt}
\titlespacing\subsection{0pt}{12pt plus 4pt minus 2pt}{0pt plus 2pt minus 2pt}
\titlespacing\subsubsection{0pt}{12pt plus 4pt minus 2pt}{0pt plus 2pt minus 2pt}


% ***********************************************************************
% *** DOCUMENT PRINCIPAL ***
% ***********************************************************************
\begin{document}
	
	\maketitle
	
	\setlength{\parskip}{1em}
	
	\section*{Questions 1 - Traitement des erreurs}
	
	Les interfaces \textit{AsyncSendRequest} et \textit{CommunicationEventListener} utilisées au point 3.1 restent très (certainement trop) simples pour être utilisables dans une vrai application : que se passe-t-il si le serveur n'est pas joignable dans l'immédiat ou s'il retourne un code HTTP d'erreur ? Vous pouvez par exemple proposer une nouvelle version de ces deux interfaces pour vous aider à illustrer votre réponse.
	
	{\color[rgb]{0,0.5,0.23}\textbf{Réponse}}
	
	Il faudrait soulever des exceptions plus précises sur le problème rencontré lors de la communication avec le serveur. Ainsi l'application principale peut traiter ces différents cas possible. \\
	Par exemple si une authentification a échouée, l'application sait qu'il faut se reconnecter et renvoyer la requête. Et comme elle sait que ce problème peut survenir, elle peut s'y préparer et le gérer. 
	
	On pourrait proposer des interfaces de ce type : 
	\begin{itemize}
		\item sendRequest
		\begin{itemize}
			\item IllegalArgumentException : les paramètres reçu ne sont pas conforme à ceux attendus par la méthode.
			\item IOException : problème de communication sur le réseau
		\end{itemize}
		\item setCommunicationEventListener
		\begin{itemize}
			\item HTTPException : pour traiter les différents cas lié au code d'erreur HTTP. Exemple : code 401 Unauthorized, lorsque le serveur nécessite une connexion.
		\end{itemize}
	
	Il serait également possible pour les HTTPException de les catch et de relever l'exception avec plus de détails, pour déjà filtrer et indiquer les cas à traiter par l'application principale.
		
	\end{itemize}
	
	\begin{lstlisting}[style=javaCode]
	public String sendRequest(String request, String url) throws IllegalArgumentException, IOException 
	public void setCommunicationEventListener (CommunicationEventListener l) 
	\end{lstlisting}
	
	\section*{Questions 2 - Authentification}
	Si une authentification par le serveur est requise, peut-ont utiliser un protocole asynchrone ? Quelles seraient les restrictions ? Peut-on utiliser une transmission différée ?
	
	
	{\color[rgb]{0,0.5,0.23}\textbf{Réponse}}
	L'authentification auprès du serveur n'empêche pas l'utilisation d'un protocole asynchrone. 
	
	
	\section*{Questions 3 - Threads concurrents}
	Lors de l'utilisation de protocoles asynchrones, c'est généralement deux threads différents qui se préoccupent de la préparation, de l'envoi, de la réception et du traitement des données. Quels problèmes cela peut-il poser ?
	
	{\color[rgb]{0,0.5,0.23}\textbf{Réponse}}
	
	Réponse
	
	\section*{Questions 4 - Écriture différée}    
	Lorsque l'on implémente l'écriture différée, il arrive que l'on ait soudainement plusieurs transmissions en attente qui deviennent possibles simultanément. Comment implémenter proprement cette situation (sans réalisation pratique) ? Voici deux possibilités :
	\begin{itemize}
		\item Effectuer une connexion par transmission différée
		\item Multiplexer toutes les connexions vers un même serveur en une seule connexion de transport. Dans ce dernier cas, comment implémenter le protocole applicatif, quels avantages peut-on espérer de ce multiplexage, et surtout, comment doit-on planifier les réponses du serveur lorsque ces dernières s'avèrent nécessaires ?
	\end{itemize}
	
	Comparer les deux techniques ( et éventuellement d'autres que vous pourriez imaginer) et discuter des avantages et inconvénients respectifs.
	
	{\color[rgb]{0,0.5,0.23}\textbf{Réponse}}
	
	Réponse
	
	
	\section*{Questions 5 - Transmission d'objets}  
	\begin{enumerate}
		\item Quel inconvénient y a-t'il à utiliser une infrastructure de type REST/JSON n'offrant aucun service de validation (DTD, XML-schéma, WSDL) par rapport à une infrastructure comme SOAP offrant ces possibilités ? Y a-t'il en revanche des avantages que vous pouvez citer ?
		\item L'utilisation d'un mécanisme comme Protocol Buffer est-elle compatible avec une architecture basée sur HTTP? Veuillez discuter des éventuelles avantages ou limitations par rapport à un protocole basé sur JSON ou XML.
	\end{enumerate} 
	
	{\color[rgb]{0,0.5,0.23}\textbf{Réponse}}
	
	Réponse
	
	\section*{Question 6 - Transmission compressée}
	Quel gain peut-on constater en moyenne sur des fichiers texte (xml et json sont aussi du texte) en utilisant de la compression du point 3.4 ? Vous comparerez vos résultats par rapport au gain théorique d'une compression DEFLATE, vous enverrez plusieurs tailles de contenu pour comparer.
	
	{\color[rgb]{0,0.5,0.23}\textbf{Réponse}}
	
	Réponse
	
	
\end{document}

